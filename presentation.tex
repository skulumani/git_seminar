%\documentclass[11pt,professionalfonts,hyperref={pdftex,pdfpagemode=none,pdfstartview=FitH}]{beamer}
%\usepackage{times}
\documentclass[11pt,professionalfonts]{beamer}
\usefonttheme{serif}
\usepackage{./presentation_packages}

\newcommand{\hilight}[1]{\colorbox{green}{#1}}

\definecolor{mygray}{gray}{0.9}
\definecolor{RoyalBlue}{rgb}{0.25,0.41,0.88}
\def\Emph{\textcolor{RoyalBlue}}

\definecolor{tmp}{rgb}{0.804,0.941,1.0}
\setbeamercolor{numerical}{fg=black,bg=tmp}
\setbeamercolor{exact}{fg=black,bg=red}

\mode<presentation> 
{
  \usetheme{Warsaw}
  \usefonttheme{serif}
  \setbeamercovered{transparent}
}

\setbeamertemplate{footline}%{split theme}
{%
  \leavevmode%
  \hbox{\begin{beamercolorbox}[wd=.5\paperwidth,ht=2.5ex,dp=1.125ex,leftskip=.3cm,rightskip=.3cm plus1fill]{author in head/foot}%
    \usebeamerfont{author in head/foot}\insertshorttitle
  \end{beamercolorbox}%
  \begin{beamercolorbox}[wd=.5\paperwidth,ht=2.5ex,dp=1.125ex,leftskip=.3cm,rightskip=.3cm]{title in head/foot}
%    \usebeamerfont{title in head/foot}\mypaper\hfill \insertframenumber/\inserttotalframenumber
    \usebeamerfont{title in head/foot}\hfill \insertframenumber/\inserttotalframenumber
  \end{beamercolorbox}}%
  \vskip0pt%
} \setbeamercolor{box}{fg=black,bg=yellow}


\title[Version Control]{\large\bf  Git to work!}

\author{\vspace*{-0.3cm}}

%\institute{\footnotesize
%{\normalsize Shankar Kulumani}\vspace*{0.2cm}\\
%  Flight Dynamics and Control Lab \\ 
%  Dept. of Mechanical and Aerospace Engineering\\ 
%  The George Washington University \\
%  Washington, DC\\ \vspace{10pt}
%  June 4, 2014 \\ \vspace{10pt}
%  }
   
\institute{
	\footnotesize
	{\normalsize\bf{Shankar Kulumani}}\\
	\vspace*{0.2cm}
  	\bf{Department of Mechanical \& Aerospace Engineering}\\ \vspace*{0.5cm}
 	\begin{figure} %figure%
       	\includegraphics[width=0.75\textwidth]{gw_txh_2cs_pos}
  	\end{figure}
}
\date{}

\begin{document}
%=======================================================%

\setcounter{framenumber}{-1}
\begin{frame} %-----------------------------%
  \titlepage
\end{frame}   %-----------------------------%

\section*{}
\subsection*{Version Control}  

\begin{frame} %-----------------------------%
\frametitle{What is version control?} % small satellites

\begin{block}{Version Control}
    A system to record or manage changes to files or sets of files over time.
\end{block}

\pause

\begin{itemize}
    \item<1-> \Emph{Collaboration} - Teams generate many variations 
    \item<2-> \Emph{Branching} - Bugs may only exist in specific versions
    \item<3-> \Emph{Trunk} - Make changes without causing more errors
\end{itemize}{}

\end{frame}   %-----------------------------%


\begin{frame}{What are some examples?}%--------------------------------------------%

\only<1>{
\begin{figure}
    \centering
    \includegraphics[height=0.75\textheight]{figures/Folder_Pizza.png}
\end{figure}
}

\only<2->{
\begin{itemize}
    \item Multiple computers or hardware (hexrotor)
    \item ROS
    \item Navy Boat
    \item Paper writing (\LaTeX)
    \item Website
\end{itemize}
}
\end{frame}%--------------------------------------------%

\begin{frame}{This is my slide}
    Shankar Kulumani
\end{frame}

\begin{frame}{Version Control Software}%--------------------------------------------%
\begin{columns}
\begin{column}[t]{0.5\textwidth}
\begin{itemize}
    \item Centralized Model
    \begin{itemize}
        \item CVS, SVN, others...
    \end{itemize}
\end{itemize}

\begin{figure}
    \centering
    \includegraphics[width=\columnwidth]{figures/centralized.png}        
\end{figure}
\end{column}

\begin{column}[t]{0.5\textwidth}
\begin{itemize}
    \item Distributed Model
    \begin{itemize}
        \item Mercurial, Git, others...
    \end{itemize}
\end{itemize}
\begin{figure}
    \centering
    \includegraphics[height=0.6\textheight]{figures/distributed.png}
\end{figure}

\end{column}
\end{columns}

\end{frame}%--------------------------------------------%

\section*{}
\subsection*{Git}

\begin{frame}{Git history}%--------------------------------------------%

Git began with a bit of creative destruction and fiery controversy...


\begin{columns}

\begin{column}[t]{0.5\textwidth}
\visible<2->{
\begin{figure}
    \includegraphics[height=\columnwidth]{figures/Linus_Torvalds.jpeg}
\end{figure}
}
\end{column}


\begin{column}[t]{0.5\textwidth}
\visible<3->{
\begin{figure}
    \includegraphics[width=0.3\textheight]{figures/Tux.pdf}\\
    \includegraphics[width=0.3\textheight]{figures/Git-logo.pdf}
\end{figure}
}
\end{column}

\end{columns}
\end{frame}%--------------------------------------------%

\begin{frame}{What is Git?} %--------------------------------------------%

\begin{itemize}
    \item<1-> Track changes in \Emph{TEXT} files!
    \item<2-> Git stores \Emph{Snapshots} - Saving the current state!
    \item<3-> Everything is local - no internet needed
    \item<4-> Integrity - Use of hash functions
\end{itemize}

\begin{figure}
    \centering
    \includegraphics<2>[width=0.75\textwidth]{figures/deltas.png}
    \includegraphics<3->[width=0.75\textwidth]{figures/snapshots.png}
\end{figure}

\end{frame}%--------------------------------------------%

\begin{frame}{Git basics}%--------------------------------------------%

\begin{enumerate}
    \item Modify files in your diretory
    \item Stage the files by adding snapshots of the current state
    \item Commit and permanently store the snapshot to the Git repo
\end{enumerate}


\begin{figure}
    \centering
    \includegraphics<1>[height=0.5\textheight]{lifecycle}
\end{figure}




\end{frame}%--------------------------------------------%

\begin{frame}{Installing Git}%--------------------------------------------%
    \url{https://git-scm.com/downloads}

    \begin{figure}
        \centering
        \includegraphics[width=0.8\textheight]{figures/download.png}
    \end{figure}
\end{frame}%--------------------------------------------%

\begin{frame}{Git Terminology}%--------------------------------------------%
\begin{itemize}
    \item Repo - Project folder that contains all the files
    \item Commit - ``Revision'' - unique change/version of a file/files
    \item Branch - Parallel version of a repo
    \item Remote - Copy of repo that lives on another computer
    \item Clone - Create a copy from a remote
    \item Push - Send your changes to a remote
    \item Fetch - Retrieve the changes from the remote
    \item Merge - Combine changes between branches
    \item Pull - Fetch and Merge changes at once
\end{itemize}
\end{frame}%--------------------------------------------%

\begin{frame}{Using Git}%--------------------------------------------%


We'll practice now!

\begin{enumerate}
    \item Working alone
    \item Working with others
    \item Git and Github
\end{enumerate}

\texttt{git pull}

\texttt{git push}
\end{frame}%--------------------------------------------%


\begin{frame}{Some helpful tips!}%--------------------------------------------%
\begin{columns}
\begin{column}{0.5\textwidth}
\begin{figure}
    \includegraphics[height=0.75\textheight]{figures/git.png}
\end{figure}
\end{column}
\begin{column}{0.5\textwidth}
\begin{itemize}
    \item Not the only/best solution!
    \item For \LaTeX: every sentence on a separate line
    \item Don't put your repo in Google Drive/Dropbox
    \item Commit often with \Emph{USEFUL} messages!
    \item If you get lost... GUI: \texttt{gitk}, \texttt{gitkraken}, or others
    \item use gitignore.io for generation of .gitignore file
\end{itemize}
\end{column}
\end{columns}
\end{frame}%--------------------------------------------%

\begin{frame}{Finished!}%--------------------------------------------%
\begin{figure}
    \centering
    \includegraphics[height=0.8\textheight]{figures/tech_support_cheat_sheet.png}
\end{figure}
\end{frame}

\begin{frame}{Useful Resources}
\begin{itemize}
    \item \href{https://git-scm.com/}{Git Book}
    \item \href{http://marklodato.github.io/visual-git-guide/index-en.html}{Visual Git}
    \item \href{http://stackoverflow.com/questions/315911/git-for-beginners-the-definitive-practical-guide}{StackOverflow}
\end{itemize}
\end{frame}%--------------------------------------------%

\end{document}

